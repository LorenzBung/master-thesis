\documentclass[10pt, a4paper, oneside]{article}
\usepackage{thesis-ufr}
\author{Lorenz Bung}
\newcommand{\topic}{Auswirkungen schulbezogener Itemformulierungen auf die gemessene Bias-Awareness von Lehrkräften}
\newcommand{\contact}{Kontakt: Lorenz Bung (\href{mailto:mail@lorenzbung.de}{\texttt{mail@lorenzbung.de}}). Vollständige Version der Arbeit: \url{https://lorenzbung.de/MA}.}

\begin{document}

\pagestyle{empty}
\pagenumbering{gobble}
\renewcommand{\bibsection}{\subsection*{Literatur}} %Rename Bibliography title
\renewcommand*{\bibfont}{\footnotesize}

\section*{\topic\footnote{\contact}}

\begin{multicols}{2}
	\subsection*{Einleitung}
	Implizite Stereotypen, also unbewusste Einflüsse auf das Denken, beeinflussen unser Leben täglich \cite{macrae2000social} und können zu negativen Erscheinungen wie selbsterfüllenden Prophezeihungen führen \cite{latu2015gender}.
	Auch im schulischen Kontext konnten diese Effekte z.B. in Bezug auf ethnischen Hintergrund \cite{glock2019studies} festgestellt werden.
	Durch eine Erhöhung der \emph{Bias Awareness (BA)}, also des Bewusstseins über eigene Biases, führt zu einer Reduktion der negativen Effekte \cite{perry2015modern}.
	Bei der Erfassung der BA spielt die Formulierung dabei eine große Rolle \cite{bing2004incremental}.
	
	\subsection*{Methodik}
	Es wurde eine quantitative Datenerhebung mit einer angepassten Version der Bias Awareness Skala \cite{perry2015modern} an n=89 Personen, davon 35 Lehramtsstudierenden und 54 Lehrkräften durchgeführt.
	Allgemeine Formulierungen sowie in Bezug auf Geschlecht, Migrationshintergrund und sozioökonomischen Status wurden jeweils mit generellen, auf SuS bezogenen und auf selbst unterrichtete SuS bezogenen Formulierungen kombiniert.
	
	\subsection*{Ergebnisse}
	Es konnte gezeigt werden, dass sowohl unter Studierenden als auch unter Lehrkräften die mit allgemein formulierter Skala erfasste BA signifikant höher war gegenüber der Erfassung mit schulspezifischer Skala (M=.463, SD=.896, d=.517, p<.001).
	Statistisch signifikante Differenzen zwischen Erfassung mit allgemeiner, schulbezogener Formulierung und in Bezug auf selbst unterrichtete SuS konnten nur bei den Studierenden festgestellt werden (M=.640, SD=1.048, p=.001, d=.610).
	Ein schwacher bis mittlerer Effekt konnte bei Vergleich allgemeiner und auf SuS formulierter Skala in Bezug auf Migrationshintergrund (M=.186, SD=.581, p=.026, d=.321), sozioökonomischen Status (M=.255, SD=.762, p<.001, d=.516) und weiblichen Personen (M=.184, SD=.669, p=.046, r=.50) unter Betrachtung der Lehrkräfte beobachtet werden.
	Bei Vergleich von Studierenden und Lehrkräften konnte kein Unterschied zwischen den Gruppen festgestellt werden.
	
	% Bibliography
	\bibliographystyle{plainnat}
	\bibliography{abstract}
\end{multicols}

\end{document}
