\chapter{Erhebung der Daten}

\section{Entwurf der Befragung}

Hypothesen:

\begin{enumerate}
	\item[H1:] \emph{Lehrkräfte haben eine höhere Bias Awareness gegenüber SuS im Vergleich zu Nicht-SuS.}
	\item[H2:] \emph{Lehrkräfte haben eine geringere Bias Awareness gegenüber SuS, die sie nicht unterrichten im Vergleich zu SuS, die von ihnen unterrichtet werden.}
	\item[H3:] \emph{Lehrkräfte haben eine geringere Bias-Awareness gegenüber Nicht-SuS im Vergleich zu SuS, die von ihnen unterrichtet werden.}
\end{enumerate}

H1 beschreibt den Unterschied der Bias Awareness zwischen SuS und Allgemeinheit, H2 den Unterschied zwischen SuS und \emph{eigenen} SuS und H3 den kombinierten Unterschied zwischen eigenen SuS und Allgemeinheit.

Die Formulierung der Fragebogenitems spielt eine entscheidende Rolle bei der Evaluierung von Konstrukten wie der Bias Awareness.
Beispielsweise kann eine kontextspezifische Formulierung der Items im Vergleich zu kontextunabhängig formulierten Items Messfehler reduzieren und die Validität erhöhen \citep{bing2004incremental}.
Durch Framing kann die Formulierung der Items die Bias Awareness beeinflussen \citep{rohner2013reducing}.
Auch das Format der Items kann eine Rolle spielen:
So reduziert eine nuancenreiche Auswahl an Antwortmöglichkeiten das Risiko für biasbehaftete Antworten \citep{elson2017question}.


\section{Durchführung der Datenerhebung}