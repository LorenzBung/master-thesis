\documentclass[11pt, a4paper]{article}
\usepackage{frontmatter}
\usepackage{background}

\begin{document}
% Title page
\thispagestyle{empty}
\backgroundsetup{scale=1, angle=0, opacity=1, contents={\includegraphics[width=\paperwidth, height=\paperheight, keepaspectratio]{resources/Vorlage_Titelblatt_Abschlussarbeit.pdf}}}
\vspace*{0.1cm}
\Huge{\textbf{\thetitle}}
%\vspace{3cm}\\
%\huge{\thetitle}
%\vspace{5cm}\\

\vspace{2cm}
\Large{vorgelegt an der \\Albert-Ludwigs-Universität Freiburg}

\vspace{2cm}
\Large{von Lorenz Bung}

\vspace{8.5cm}
\dateend
\newpage
\backgroundsetup{contents={}}

\thispagestyle{empty}
{
	\setlength{\parskip}{1cm}
	\begin{center}
		\vspace*{1cm}
		\textbf{\Huge MASTERARBEIT}
		\vspace{2cm}\\
		
		\textbf{zur Erlangung des akademischen Grades}
		
		\textbf{\Large Master of Education (M. Ed.)}
		
		\textbf{an der}
		
		{\huge Albert-Ludwigs-Universität Freiburg}\\
		
		{\Large Wirtschafts- und Verhaltenswissenschaftliche Fakultät}
		\vspace{1cm}
		
		\renewcommand{\arraystretch}{1.0}	%Zeilenabstand in der Tabelle reduzieren
		\begin{tabular}{p{4cm} p{10cm}}
			Thema: & \thetitle \\&\\
			Kandidat: & \theauthor \\
			& Schlossbergring 36 \\
			& 79098 Freiburg \\&\\
			Prüfer: & Prof. Dr. Meike Bonefeld \\
			Ausgabedatum: & \datestart \\
			Abgabedatum: & \dateend \\
		\end{tabular}
		\renewcommand{\arraystretch}{1.5}	%Zeilenabstand in der Tabelle zurücksetzen
	\end{center}
}
\newpage

\thispagestyle{empty}
\tableofcontents
\newpage
\setcounter{page}{1}


\section{Einleitung}

In vielen Bereichen unseres Lebens sind wir - oft unbewusst - zahlreichen kognitiven Verzerrungen ausgesetzt \cite{pohl2004cognitive}.
Dass sich diese Biases auch auf das Handeln von Lehrkräften im Klassenraum auswirken, ist schon lange bekannt - beispielsweise können die Erwartungen von Lehrkräften an ihre Schülerinnen und Schüler (SuS) die Leistungen dieser beeinflussen \cite{rosenthal1968pygmalion}.

Eine mögliche Erklärung für diesen Effekt ist die Stigmatisierung von Randgruppen durch die Lehrkräfte - eine Tatsache, die bereits durch mehrere Studien belegt wurde \cite{glock2015preservice, glock2017bad, glock2019studies}.
Insbesondere negative Erwartungen der Lehrkräfte haben starke Einflüsse auf die Erfolgschancen der SuS \cite{jussim1996social}, was häufig in sich ``selbsterfüllenden Prophezeihungen'' resultiert \cite{jussim2005teacher}.
Beispielsweise haben viele SuS mit niedrigem sozio-ökonomischem Status geringere Erwartungen an sich selbst, als ihre akademischen Leistungen erwarten lassen würden \cite[S.~15]{oecd2019}.


\section{Zielsetzung}

Vor dem Hintergrund der vorliegenden Biases von Lehrkräften stellt sich die Leitfrage, ob die persönlichen Beziehungen der Lehrkräfte zu den eigenen SuS einen Einfluss auf ihre Bias Awareness haben.
Es soll mittels einer Datenerhebung analysiert werden, ob sich Unterschiede in der Bias Awareness von Lehrkräften bei ihren eigenen Schülerinnen und Schülern im Vergleich zur Allgemeinheit ergeben und wie groß deren Ausprägung ist.
Ziel der Arbeit sind folgende Punkte:
\begin{enumerate}
	\item Entwurf einer geeigneten Datenerhebung zur Feststellung der Ausprägung oben genannter Aspekte
	\item Durchführung der Erhebung
	\item Verarbeitung, Auswertung und Analyse der Daten
	\item Interpretation und Diskussion der Ergebnisse hinsichtlich der Leitfrage
\end{enumerate}


\section{Methodik}

Um eine geeignete Datenerhebung durchführen zu können, muss zunächst ein fachliches Fundament hergestellt werden.
Hierzu wird eine umfassende Literaturrecherche den aktuellen Stand der Forschung erfassen, auf Grund dessen anschließend eine Umfrage durchgeführt werden kann.
Zusätzlich werden in diesem Teil erforderliche qualitative Kriterien für die empirische Forschung festgehalten, welche als Basis für die Umfrage dienen.

Zur Beantwortung der Forschungsfrage werden zunächst Hypothesen aufgestellt, welche anschließend mithilfe der Datenerhebung überprüft werden.
Im Zentrum steht dabei folgende Unterschiedshypothese:
\begin{quote}
	\itshape
	Lehrkräfte haben eine geringere Bias Awareness gegenüber Personen, die sie nicht unterrichten im Vergleich zu Personen, die von ihnen unterrichtet werden.
\end{quote}

Die Datenerhebung erfolgt in Form einer Umfrage unter Lehrkräften.
Ausgewertet werden die Daten mithilfe deskriptiver Statistik und Inferenzstatistik.
Auf Basis der hier erlangten Ergebnisse werden die aufgestellten Hypothesen bewertet.

Im anschließenden Diskussionsteil werden die Resultate eingeordnet, um danach in Zusammenfassung und Fazit in eine abschließende Bewertung einzufließen.


\section{Vorläufige Gliederung}

\begin{enumerate}
	\item Einleitung
	\begin{enumerate}[label*=\arabic*.]
		\item Motivation
		\item Zielsetzung
	\end{enumerate}
	\item Hintergrund und bisherige Forschung \label{bisherige-forschung}
	\item Erhebung der Daten \label{datenerhebung}
	\begin{enumerate}[label*=\arabic*.]
		\item Qualitätskriterien empirischer Forschung
		\item Entwurf der Befragung
		\item Durchführung der Datenerhebung
	\end{enumerate}
	\item Auswertung \label{auswertung}
	\begin{enumerate}[label*=\arabic*.]
		\item Datenaufbereitung
		\item Statistische Analyse
	\end{enumerate}
	\item Diskussion
	\item Fazit und Zusammenfassung
\end{enumerate}


\section{Zeitplanung}

Die in der Studienordnung des Master of Education vorgegebene Bearbeitungszeit für die Masterarbeit beträgt 4 Monate \cite[\S19.3]{unifreiburg2018studien}.
Der angegebene Anmeldezeitpunkt kann sich je nach organisatorischem Aufwand und Praktikabilität noch verändern.
Der Einfachheit halber wird jedoch vom 01.09. ausgegangen; sämtliche in \autoref{tab:zeitplanung} angegebenen Daten beziehen sich darauf.

\begin{table}[h]
\begin{tabularx}{\textwidth}{|X|X|}
	\hline
	\textbf{Zeitraum} & \textbf{Geplante Tätigkeit}\\
	\hline\hline
	01.09.2024 & Anmeldung der Masterarbeit\\
	\hline
	01.09.2024 - 15.09.2024 & Literatursichtung\\
	& Erster grundlegender Entwurf des \hyperref[bisherige-forschung]{Theorieteils} (Gliederungspunkt 2)\\
	\hline
	16.09.2024 - 29.09.2024 & Studienplanung\\
	\hline
	30.09.2024 - 03.11.2024 & Durchführung der Studie\\
	& Überarbeitung des \hyperref[bisherige-forschung]{Theorieteils}\\
	& Entwurf \hyperref[datenerhebung]{Datenerhebung} (Gliederungspunkt 3)\\
	\hline
	04.11.2024 - 24.11.2024 & Auswertung der Daten\\
	\hline
	25.11.2024 - 08.12.2024 & Dokumentation der \hyperref[auswertung]{Datenauswertung}\\
	& Entwurf Diskussion, Fazit und Zusammenfassung\\
	& Entwurf Einleitung\\
	\hline
	09.12.2024 - 22.12.2024 & Finale Überarbeitungen\\
	& Korrekturen und Verbesserungen durch Probelesende\\
	\hline
	23.12.2024 - 30.12.2024 & Druck der Arbeit\\
	& Puffer\\
	\hline
	31.12.2024 & Abgabe der Masterarbeit\\
	\hline
\end{tabularx}
\caption{Zeitplanung für die Masterarbeit}
\label{tab:zeitplanung}
\end{table}

% Bibliography
\newpage
\bibliographystyle{plain}
\bibliography{expose}
\end{document}